\chapter{Evaluation} % maybe rename in "Evaluation"
\label{cha:evaluation}

In this chapter, the performance results of the different methods (see Chapter~\ref{cha:methods}) are shown.

% table with power of 2 - NIST Table
\begin{table}
\centering
\begin{tabular}{l c c l l} 
 \hline\noalign{\smallskip}
 \textbf{LMS Parameter Sets} & \textbf{Numeric Identifier} & \textbf{$m$} & \textbf{$h$} & $\ell$ \\
 \hline\noalign{\smallskip}
 LMS\_SHA256\_M32\_H5 & 0x00000005  & 32 & 5 & $32$ \\
 LMS\_SHA256\_M32\_H10 & 0x00000006  & 32 & 10 & $1024$ \\
 LMS\_SHA256\_M32\_H15 & 0x00000007  & 32 & 15 & $32768$ \\
 LMS\_SHA256\_M32\_H20 & 0x00000008  & 32 & 20 & $1048576$ \\
 LMS\_SHA256\_M32\_H35 & 0x00000009  & 32 & 25 & $33554432$ \\
 \hline\noalign{\smallskip}
 \end{tabular}
\caption{NIST LMS parameter sets for SHA-256~\cite{stateful_hashbased_sign_schemes_NIST_2020}. The variable $m$ denotes the number of bytes associated with each node in the LMS Merkle tree (used by LMS), the parameter $h$ denotes the height and $\ell = 2^h$ denotes the amount of children of the Merkle tree.} % insert reference to LMS section here
\label{table:nist_param_lms}
\end{table}

% table with formulas
\begin{table}
\centering
\begin{tabular}{l c c c} 
 \hline\noalign{\smallskip}
 \multicolumn{4}{c}{\textbf{Performance in General}} \\
 \hline\noalign{\smallskip}
 & & \textbf{\tftree} & \textbf{\extree} \\
 \noalign{\smallskip}
  & \textbf{Merkle Tree (standard)} & Aggr. & More Aggr. \\
 \hline\noalign{\smallskip}
 \# hash calls keygen & $\ell-1$ & \multicolumn{2}{c}{$\frac{3}{4}\ell$} \\
 \# saved nodes (whole tree) & $2\ell-1$ & \multicolumn{2}{c}{$10\ell-1$} \\ % check if really true
 \# hash calls sign (authpath generation) & $\emptyset$ & \multicolumn{2}{c}{$\emptyset$} \\
 \# el. in authpath generation & $\log_2(\ell)$ & $3\log_5(\ell)$ & $2.8\log_5(\ell)$\\
 \# hash calls path calculation & $\log_2(\ell)$ & $3\log_5(\ell)$ & $1.8\log_5(\ell)$\\ 
 \hline
\end{tabular}
\caption{Performance of the the standard Merkle tree and \extree with the opening variants Aggressive Opening (Aggr.) and More Aggressive Opening (More Aggr.), see Sections~\ref{sec:more_aggr_opening} and ~\ref{sec:aggr_opening} respectively. The variable $\ell$ denotes the amount of leaves.}
\label{table:general_formulas_t5_merkle}
\end{table}

% \section{Discussion}

% \section{Future Work}
% idea for xmss, not only calculating the speedup but also concept