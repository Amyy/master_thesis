\chapter{Evaluation}
\label{cha:evaluation}
In Chapter~\ref{cha:background}, the two most common quantum secure hash-based signature systems LMS and XMSS based on binary Merkle Trees are explained. In Chapter~\ref{cha:methods}, the concepts \tftree (see Section~\ref{sec:dodis_t5_merkle_tree}) and \extree (see Section~\ref{sec:ext_t5_tree}) are proposed in order to speed up the common binary Merkle Tree and therefore the signature systems LMS and XMSS. 
% ! xmss not so easy possible -> bitmasks!

\section{NIST Parameter Sets}
In order to compare the different concepts, the performance is calculated for the SHA-256 NIST parameter set for LMS~\cite{stateful_hashbased_sign_schemes_NIST_2020}, see Table~\ref{table:nist_param_lms}. 

\begin{table}
\centering
\begin{tabular}{l c c l l} 
 \hline\noalign{\smallskip}
 \textbf{LMS Parameter Sets} & \textbf{Numeric Identifier} & \textbf{$m$} & \textbf{$h$} & $\ell$ \\
 \hline\noalign{\smallskip}
 LMS\_SHA256\_M32\_H5 & 0x00000005  & 32 & 5 & $32$ \\
 LMS\_SHA256\_M32\_H10 & 0x00000006  & 32 & 10 & $1024$ \\
 LMS\_SHA256\_M32\_H15 & 0x00000007  & 32 & 15 & $32768$ \\
 LMS\_SHA256\_M32\_H20 & 0x00000008  & 32 & 20 & $1048576$ \\
 LMS\_SHA256\_M32\_H35 & 0x00000009  & 32 & 25 & $33554432$ \\
 \hline\noalign{\smallskip}
 \end{tabular}
\caption{NIST LMS parameter sets for SHA-256~\cite{stateful_hashbased_sign_schemes_NIST_2020}. The variable $m$ denotes the number of bytes associated with each node in the LMS Merkle tree (used by LMS), the parameter $h$ denotes the height and $\ell = 2^h$ denotes the amount of children of the Merkle tree.} % insert reference to LMS section here
\label{table:nist_param_lms}
\end{table}

An overview of the equations for performance calculation introduced in Chapter~\ref{cha:methods} is given in Table~\ref{table:general_formulas_t5_merkle}.
\begin{table}
\centering
\begin{tabular}{l c c c l} 
 \hline\noalign{\smallskip}
 \multicolumn{5}{c}{\textbf{Summary Equations for Performance Calculation}} \\
 \hline\noalign{\smallskip}
 & & \textbf{\tftree} & \textbf{\extree} & \\
 \noalign{\smallskip}
  & \textbf{Merkle Tree} & Aggr. & More Aggr. & \textbf{Source} \\
 \hline\noalign{\smallskip}
 \# hash calls keygen & $\ell-1$ & \multicolumn{2}{c}{$\frac{3}{4}(\ell-1)$} & Eq.~\ref{eq:lms_hashcalls_tree_treegen}, \ref{eq:t5_tree_gen_hashcalls} \\
% \# saved nodes (whole tree) & $2\ell-1$ & \multicolumn{2}{c}{$10\ell-1$} \\ % check if really true
 \# hash calls sign & $\emptyset$ & \multicolumn{2}{c}{$\emptyset$} & \\ % auth.path generation
 \# el. in auth.path & $1.44\log(\ell) $ & $1.86\log(\ell)$ & $1.74\log(\ell)$ & Eq. \ref{eq:lms_authpath_el}, \ref{eq:t5_el_authpath}, \ref{eq:ext_t5_len_authpath} \\
 \# hash calls verify & $1.44\log(\ell)$ & $1.24\log(\ell)$ & $1.12\log(\ell)$ & Eq. \ref{eq:lms_hashcalls_verify}, \ref{eq:t5_path_calc_hashcalls}, \ref{eq:ext_t5_hashcalls_verify} \\  % path calculation
 \hline
\end{tabular}
\caption{Performance of the the standard Merkle Tree (used in LMS) and \extree with the opening variants Aggressive Opening (Aggr.) and More Aggressive Opening (More Aggr.), see Sections~\ref{sec:more_aggr_opening} and \ref{sec:aggr_opening} respectively. The variable $\ell$ denotes the amount of leaves.}
\label{table:general_formulas_t5_merkle}
\end{table}

% \section{Discussion}

% \section{Future Work}
% idea for xmss, not only calculating the speedup but also concept