\chapter{Evaluation}
\label{cha:evaluation}
In Chapter~\ref{cha:background}, the two most common quantum secure hash-based signature systems based on binary Merkle Trees are explained: LMS (see Section~\ref{sec:lms}) and XMSS (see Section~\ref{sec:xmss}). In Chapter~\ref{cha:methods}, the concepts \tftree (see Section~\ref{sec:dodis_t5_merkle_tree}) and \extree (see Section~\ref{sec:ext_t5_tree}) are proposed in order to speed up the common binary Merkle Tree and therefore the signature systems LMS and XMSS. 
% ! xmss not so easy possible -> bitmasks!

\section{Performance Comparison}
The general performance of the different tree concepts depending on the leaves $\ell$ is calculated with the equations in Table~\ref{table:general_formulas_t5_merkle}. The derivation of each formula is referenced in the column \textit{Source}.
Notably, these formulas do not take into consideration that, depending on the tree concept, the leaves have a power of two or five. Still, they state a general speedup of each tree concept regarding tree generation, length of authentication path and verification. The general speedup is denoted in Table~\ref{table:perform_differences}.

\begin{table}
\centering
\begin{tabular}{l c c c l} 
 \hline\noalign{\smallskip}
 \multicolumn{5}{c}{\textbf{Summary: Equations Performance Calculation}} \\
\hline\noalign{\smallskip}
 & Merkle Tree & \tftree & \extree & Source  \\
 \noalign{\smallskip}
  &  & Aggr. & More Aggr. & \\
 \hline\noalign{\smallskip}
 \# hash calls keygen & $\ell-1$ & \multicolumn{2}{c}{$\frac{3}{4}(\ell-1)$} & Eq.~\ref{eq:lms_hashcalls_tree_treegen}, \ref{eq:t5_tree_gen_hashcalls} \\
 \# hash calls verify & $1.44\log(\ell)$ & $1.24\log(\ell)$ & $1.12\log(\ell)$ & Eq. \ref{eq:lms_hashcalls_verify}, \ref{eq:t5_path_calc_hashcalls}, \ref{eq:ext_t5_hashcalls_verify} \\ 
\hline\noalign{\smallskip}
 \# el. in auth.path & $1.44\log(\ell) $ & $1.86\log(\ell)$ & $1.74\log(\ell)$ & Eq. \ref{eq:lms_authpath_el}, \ref{eq:t5_el_authpath}, \ref{eq:ext_t5_len_authpath} \\
 \hline
\end{tabular}
\caption{Performance of the the standard Merkle Tree (used in LMS) and \extree with the opening variants Aggressive Opening (Aggr.) and More Aggressive Opening (More Aggr.), see Sections~\ref{sec:more_aggr_opening} and \ref{sec:aggr_opening} respectively. The variable $\ell$ denotes the amount of leaves.}
\label{table:general_formulas_t5_merkle}
\end{table}

\begin{table}
\centering
\begin{tabular}{l c c } 
 \hline\noalign{\smallskip}
 \multicolumn{3}{c}{\textbf{General Performance Comparison}} \\
 \noalign{\smallskip}
  & \tftree & \extree \\
 \hline\noalign{\smallskip}
 hash calls: key gen. & \multicolumn{2}{c}{$-25\%$} \\
 hash calls: verify & $-14\%$ & $-22\%$ \\ 
 \hline\noalign{\smallskip}
 length auth. path & $+ 29\%$ & $+ 20\%$ \\
 \hline
\end{tabular}
\caption{Performance of \tftree and \extree in comparison to the Merkle Tree:  The amount of hash calls for tree generation and verification decreases for \tftree and \extree respectively, while the length of the authentication path increases.} 
\label{table:perform_differences}
\end{table}

\section{NIST Parameter Set}
There exist standardized sets of values that are used for the proposed signature systems. One common example is the LMS SHA-256 parameter set by the National Institute of Standards and Technology (NIST)~\cite{stateful_hashbased_sign_schemes_NIST_2020}: It is denoted in Table~\ref{table:nist_param_lms}. 

\begin{table}
\centering
\begin{tabular}{l c c c l} 
 \hline\noalign{\smallskip}
 \multicolumn{5}{c}{\textbf{NIST Parameter Set, LMS}} \\
 Parameter Set Name & Numeric Identifier & $n$ & $d$ & $\ell$\\
 \hline\noalign{\smallskip}
 LMS\_SHA256\_M32\_H5 & 0x00000005  & 32 & 5 & 32 \\
 LMS\_SHA256\_M32\_H10 & 0x00000006  & 32 & 10 & 1024 \\
 LMS\_SHA256\_M32\_H15 & 0x00000007  & 32 & 15 & 32768 \\
 LMS\_SHA256\_M32\_H20 & 0x00000008  & 32 & 20 & 1048576 \\
 LMS\_SHA256\_M32\_H35 & 0x00000009  & 32 & 25 & 33554432 \\
 \hline\noalign{\smallskip}
 \end{tabular}
\caption{NIST SHA-256 parameter sets for LMS.~\cite{stateful_hashbased_sign_schemes_NIST_2020}. The variable $n$ denotes the number of bytes associated with each node in the (standard binary) Merkle tree, the parameter $d$ denotes the height and the parameter $\ell$ the leaves of the Merkle Tree.}
\label{table:nist_param_lms}
\end{table}

\subsection{NIST Parameter Adaption}
\label{sec:nist_param_to_leaves}
When used as digital signature scheme, the leaves of the Merkle Tree, \tftree and \extree correspond to the amount of one-time keys. For this evaluation, we assume each leaf contains a one-time public key (i.e. there are no empty nodes). When comparing the performance of the standard Merkle Tree with \tftree and \extree based on the amount of leaves $\ell$, it is not possible to get the same amount of leaves for each concept, because the leaves of a perfect Merkle Tree are always a power of two, whereas the leaves of a perfect \tftree\xspace/\xspace\extree are always a power of five. 
 
In order to still get a similar amount of leaves, all $2^d, d \in \{5,10,15,20,25\}$ are paired with their lower and upper closest power of 5:
These \textit{upper bounds} and \textit{lower bounds} for a given $2^d$ are calculated by $5^{\floor{\log_5(2^d)}}$ and $5^{\ceil{\log_5(2^d)}}$ respectively. The results are shown in Table~\ref{table:nist_upper_lower_bound}. 

\begin{table}
\centering
\begin{tabular}{r c l c c} 
 \hline\noalign{\smallskip}
 \multicolumn{5}{c}{\textbf{Evaluation Results: Lower / Upper Bound \tftree\xspace/\xspace\extree}} \\
 \noalign{\smallskip} 
 & lower / upper  & Tree & Auth.path Length & Verify \\
 \noalign{\smallskip}
 & bound: $\ell$  & Generation & aggr. / more aggr. & aggr. / more aggr.\\
 \hline\noalign{\smallskip}
 \multirow{2}{*}{$2^5$} & $\rightarrow 5^{\floor{\log_5(2^5)}} = 5^{2} $ & 18 & 6 / 6 & 4 / 4 \\
 & $\rightarrow 5^{\ceil{\log_5(2^5)}} = 5^{3}$ & 93 & 9 / 8 & 6 / 5 \\
 \hline\noalign{\smallskip} 
 \multirow{2}{*}{$2^{10}$} & $\rightarrow 5^{\floor{\log_5(2^{10})}} = 5^{4}$ & 468 & 12 / 11 & 8 / 7 \\
 & $\rightarrow 5^{\ceil{\log_5(2^{10})}} = 5^{5}$ & 2343 & 15 / 14 & 10 / 9 \\
 \hline\noalign{\smallskip} 
 \multirow{2}{*}{$2^{15}$}& $\rightarrow 5^{\floor{\log_5(2^{15})}} = 5^{6}$ & 11718 & 18 / 17 & 12 / 11 \\ 
 & $\rightarrow 5^{\ceil{\log_5(2^{15})}} = 5^{7}$ & 58593 & 21 / 20 & 14 / 13 \\ 
 \hline\noalign{\smallskip} 
 \multirow{2}{*}{$2^{20}$} & $\rightarrow 5^{\floor{\log_5(2^{20})}} = 5^{8}$ & 292968 & 24 / 22 & 16 / 14 \\ 
 & $\rightarrow 5^{\ceil{\log_5(2^{20})}} = 5^{9}$ & 1464843 & 27 / 25 & 18 / 16 \\
 \hline\noalign{\smallskip}  
  \multirow{2}{*}{$2^{25}$} & $\rightarrow 5^{\floor{\log_5(2^{25})}} =  5^{10}$ & 7324218 & 30 / 28 & 20 / 18 \\ 
 & $\rightarrow 5^{\ceil{\log_5(2^{25})}} = 5^{11}$ & 36621093 & 33 / 31 & 22 / 20 \\
 \hline\noalign{\smallskip}
 \end{tabular}
\caption{Correlation between Merkle leaves $\ell$ (given by the NIST SHA-256 parameter set, see Table~\ref{table:nist_param_lms}) and leaves of the corresponding upper / lower bound \tftree\xspace/\xspace\extree and the evaluation results for tree generation, length of the authentication path and path generation (verify).}
\label{table:nist_upper_lower_bound}
\end{table}


\subsection{NIST Parameter Results}
After determining the parameters for each tree concept in the section before, the parameters are inserted into the equations for evaluation (see Table~\ref{table:general_formulas_t5_merkle}) to get tangible results. The performance calculation is also implemented in a Python script, see Appendix~\ref{cha:appendix2_performance_calc}.
The evaluation results for the Merkle Tree are shown in Table~\ref{table:eval_merkle_tree_NIST}, for the lower/upper bound \tftree and \extree in Table~\ref{table:nist_upper_lower_bound}.

\begin{table}
\centering
\begin{tabular}{c l c c} 
 \hline\noalign{\smallskip}
 \multicolumn{4}{c}{\textbf{Evaluation Results NIST: Merkle Tree}} \\
 \noalign{\smallskip} 
  Leaves $\ell$ & Tree Generation & Auth.path Length & Verify \\
% $\ell$ & (in hash calls) & (in elements) & (in hash calls) \\
 \hline\noalign{\smallskip}
 $2^5$ & 31 & 5 & 5 \\
 $2^{10}$ & 1023 & 10 & 10 \\
 $2^{15}$ & 32767 & 15 & 15 \\ 
 $2^{20}$ & 1048575 & 20 & 20 \\ 
 $2^{25}$ & 33554431 & 25 & 25 \\ 
 \hline\noalign{\smallskip}
 \end{tabular}
\caption{Evaluation results for the standard Merkle Tree, given the NIST SHA-256 parameter sets as leaves $\ell$ (see Table~\ref{table:nist_param_lms}). The results contain the number of hash calls for tree generation and verify, as well as the length of the authentication path given the leaves $\ell$.}
\label{table:eval_merkle_tree_NIST}
\end{table}

The effort for tree generation for each tree concept is depicted in Figure~\ref{img:performance_tree_gen}.

\begin{figure}
\centering
\includegraphics[width=\linewidth]{images/Evaluation/performance_tree_generation.png}
\caption{Amount of hash calls for tree generation of each concept. Notably, \tftree and \extree have the same performance for tree calculation, therefore they are not mentioned separately here. The height $d$ of the Merkle Tree is directly based on the NIST parameter set (see Table~\ref{table:nist_param_lms}). The height $d$ of the upper- and lower bound \tftree is derived from the NIST parameter set, see Table~\ref{table:nist_upper_lower_bound}. For equations used for performance calculation, see Table~\ref{table:general_formulas_t5_merkle}.}
\label{img:performance_tree_gen}
\end{figure}

