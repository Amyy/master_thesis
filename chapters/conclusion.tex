\chapter{Conclusion}
\label{cha:conclusion}

In this work, the goal was to speed up common hash-based signature systems.

\section{Discussion}

% it is possible to insert T5 idea into LMS. 
% Hashcalls winternitz to get to leaves are not taken into account in this work, because 

\section{Future Work} % Future work: belongs here

% t5 idea is possible in xmss: XMSS inspected in detail in this work, therefore -> would work with t5 but take into account bitmasks. Therefore, t5 tree concept will have to be adapted.

% It is possible to adopt the \extree concept into SPHINCS simple, because SPHINCS simple does not contain bitmasks. Would work like LMS+T5 concept.

% Concept for T5 tree when not every leaf is "filled" ? -> Possible to use a more variants of one-time key sets, because NIST Parameter set (reference here) fits to binary tree and not leaves = power of 5 tree.
% Idea: do not calculate all "subtrees" of T5 tree.

% Tradeoff: what the signer/verifier generates in one T5 block. For example, signer gives verifier c,d instead of a,b -> signer has to calculate 2 XORs, not verifier or vise versa.