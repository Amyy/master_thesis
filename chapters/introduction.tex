\chapter{Introduction}
\label{cha:introduction}
In this chapter, the motivation and goals of this work are presented. First, the quantum threat is introduced. Afterwards, the goals of this work are defined. The last section describes the general structure of this thesis.

\section{Quantum Threat} % Motivation
Quantum computing theory has been researched extensively and is considered the greatest threat to modern cryptography, also referred to as \textit{quantum threat}~\cite{impact_quantum_crypto_2018}.
It was first mentioned in 1996, when Shor~\cite{shors_algo_original_1999} proposed a quantum algorithm for factorization that is exponentially faster than any known classical algorithm. In 1996, Grover~\cite{grovers_algo_basispaper_1997} proposed a quantum searching algorithm, speeding up search efficiency from classical $\mathcal{O}(N)$ to $\sqrt{N}$. 
Based on these discoveries, further research shows that present asymmetric cryptographic schemes (whose security is based on the difficulty of factorizing large prime numbers and the discrete logarithm problem) can be broken, once a quantum computer with a sufficient number of quantum bits exists. 
Symmetric cryptography is also effected by this quantum threat. However, its security can be increased by using larger key spaces.~\cite{impact_quantum_crypto_2018} % also basic quantum comp. paper here
Example quantum attacks on RSA~\cite{rsa_patent} (whose security relies on the prime factorization problem) are shown by Soni et al.~\cite{RSA_pq-attack_examples_2018} and Wang et al.~\cite{RSA_pq-attack_without_factorization_2018}.

There are some classes of cryptographic systems considered more resistant against quantum attacks, also denoted as \textit{quantum secure}: Hash-based cryptography, code-based cryptography, lattice-based cryptography, multivariate-quadratic-equations cryptography and symmetric cryptography.~\cite{book_pqc_bernstein_2004}
This work focuses on hash-based cryptography used for digital signature systems, referred to as \textit{hash-based signature systems (HBS)}.

\section{Goals} 
Quantum secure cryptographic schemes have an overall worse performance in comparison to classical cryptographic systems. % source ?
This is also a problem for hash-based signature systems. The classical algorithms elliptic curve digital signature algorithm (ECDSA)~\cite{ecdsa_main_paper_2001} and Rivest-Shamir-Adleman (RSA~\cite{rsa_patent}) outperform HBS. 
Noel et al.~\cite{comparison_performance_RSA_ECDSA_Merkle_WOTS_2021} show that these two classical algorithms outperform the Merkle Signature Scheme (common HBS, see Section~\ref{sec:mss}) in key generation, signature generation and verification time.
Therefore, finding possible efficiency improvements for existing HBS is a crucial task and also the main goal of this work.
For related work on improving the performance of HBS in several ways, see Chapter~\ref{cha:stateOfTheArt}.

\section{Structure}
In Chapter~\ref{cha:background} the fundamentals necessary for this work are introduced: After a brief introduction of digital signature schemes in general, the one-time signature schemes Lamport-Diffie One-Time Signature Scheme and Winternitz One-Time Signature Scheme are presented, including a basic example. Afterwards, the Merkle Signature Scheme, Leighton-Micali Signature Scheme and eXtended Merkle Signature Scheme are explained in detail. Chapter~\ref{cha:stateOfTheArt} summarizes state-of-the-art literature related to the thesis topic. 
Chapter~\ref{cha:methods} proposes methods to improve the performance of HBS.
Afterwards, these methods are evaluated in Chapter~\ref{cha:evaluation}.
Finally, Chapter~\ref{cha:conclusion} discusses the results of this work and provides possibilities for future work.
In Appendix~\ref{cha:appendix_t5_tree_implementation} and Appendix~\ref{cha:appendix2_performance_calc}, the \extree concept and generation of the evaluation results are implemented.