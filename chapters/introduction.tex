\chapter{Introduction}
\label{cha:introduction}
In this chapter, the motivation for this work, the quantum threat and an introduction into the topic HBS is given.

\section{Quantum Threat}
% maybe: what is a quantum computer / quantum computing here
Quantum computing theory has been researched extensively and is considered the greatest threat to present modern cryptography~\cite{impact_quantum_crypto_2018}.
The beginning was in 1996, when Shor~\cite{shors_algo_original_1999} proposed a quantum algorithm for factorization that is exponentially faster than any known classical algorithm. In 1996, Grover~\cite{grovers_algo_basispaper_1997} proposed a quantum searching algorithm, speeding up search efficiency from classical $\mathcal{O}(N)$ to $\sqrt{N}$. 
Based on these discoveries, further research shows that present asymmetric cryptographic schemes (whose security is based on the difficulty of factorizing large prime numbers and the discrete logarithm problem) can be broken, once a quantum computer with a sufficient number of qubits exists. 
Symmetric cryptography is also effected by this quantum threat, however, its security can be increased by using larger key spaces.
~\cite{impact_quantum_crypto_2018} % also basic quantum comp. paper here
Example quantum attacks on RSA, which security relies on the prime factorization problem, is shown by Soni et al.~\cite{RSA_pq-attack_examples_2018} and Wang et al.~\cite{RSA_pq-attack_without_factorization_2018}.

% ECC Attack examples -> probably unnecessary

% https://globalriskinstitute.org/publications/quantum-threat-timeline-report-2020/ -> nice timeline about quantum threat 

% \section{Motivation} / Goals, one section?
% defeat quantum-threat -> good source: https://dl.acm.org/doi/fullHtml/10.1145/3398388

% performance PQC Algos -> worse than "classic" ones
%Comparison RSA/ECDSA with WOTS/MSS:
%"The results showed that the two classical algorithms perform better in terms of the efficiency in key generation time, signature generation and verification time."~\cite{comparison_performance_RSA_ECDSA_Merkle_WOTS_2021}

% -> improve performance of PQC algorithm

% -> look at HBS in this work

\section{Goals} 
% Finding possible improvements to existing hashbased signature systems / make them more efficient / faster / smaller signature size.
% other works that make existing hash-based signature more efficient: See state of the art

% T5 Paper hier bei Zielen -> Ziel ist Aufwand von Verfahren zu reduzieren, ohne Sicherheit zu beeinflussen

% Reference to Methods/T5 stuff etc.
The goal is to substitute the standard Merkle Tree with the \tftree and \extree concepts, to speed up LMS, XMSS and potentially other signature schemes based on Merkle Trees.


% other variants (maybe introduction):
% picnic -> ZKP scheme?
% Rainbow, GeMMs -> MultiVariate
% crystals-dilithium, falcon -> Lattice
% BPQS = blockchained post quantum signatures -> https://eprint.iacr.org/2018/658.pdf

% maybe add examples/cases where Merkle tree outperforms RSA/ECDSA. -> https://link.springer.com/chapter/10.1007/978-3-540-85893-5_8 , year 2008

%The common classes of quantum-secure cryptographic systems are as follows:~\cite{book_pqc_bernstein_2004}
%\begin{enumerate}
%\item Hash-based cryptography
%\item Code-based cryptography
%\item Lattice-based cryptography
%\item Multivariate-quadratic-equations cryptography
%\item Secret-key cryptography
%\end{enumerate}