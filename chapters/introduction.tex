\chapter{Introduction}
\label{cha:introduction}
In this chapter, the motivation and goals of this work are proposed. First, the quantum threat is introduced. Afterwards, the goals of this work are proposed. The last section describes the general structure of this thesis.

\section{Quantum Threat} % Motivation
Quantum computing theory has been researched extensively and is considered the greatest threat to present modern cryptography, also referred to as \textit{quantum threat}~\cite{impact_quantum_crypto_2018}.
The beginning was in 1996, when Shor~\cite{shors_algo_original_1999} proposed a quantum algorithm for factorization that is exponentially faster than any known classical algorithm. In 1996, Grover~\cite{grovers_algo_basispaper_1997} proposed a quantum searching algorithm, speeding up search efficiency from classical $\mathcal{O}(N)$ to $\sqrt{N}$. 
Based on these discoveries, further research shows that present asymmetric cryptographic schemes (whose security is based on the difficulty of factorizing large prime numbers and the discrete logarithm problem) can be broken, once a quantum computer with a sufficient number of qbits exists. 
Symmetric cryptography is also effected by this quantum threat, however, its security can be increased by using larger key spaces.~\cite{impact_quantum_crypto_2018} % also basic quantum comp. paper here
Example quantum attacks on RSA~\cite{rsa_patent} (which security relies on the prime factorization problem) are shown by Soni et al.~\cite{RSA_pq-attack_examples_2018} and Wang et al.~\cite{RSA_pq-attack_without_factorization_2018}.

There exist five main classes of cryptographic systems considered more resistant against quantum attacks, also denoted as \textit{quantum secure}: Hash-based cryptography, code-based cryptography, lattice-based cryptography, multivariate-quadratic-equations cryptography and secret-key cryptography.~\cite{book_pqc_bernstein_2004}
In this work, the focus lies on hash-based cryptography used for digital signature systems, referred to as \textit{hash-based signature systems (HBS)}.

\section{Goals} 
Quantum secure cryptographic schemes have an overall worse performance in comparison to classical cryptographic systems, % source ?
this is also a problem for hash-based signature systems. The classical algorithms elliptic curve digital signature algorithm (ECDSA) and Rivest-Shamir-Adleman (RSA) outperform HBS. Noel et al.~\cite{comparison_performance_RSA_ECDSA_Merkle_WOTS_2021} show that the two classical algorithms ECDSA (elliptic curve digital signature algorithm~\cite{ecdsa_main_paper_2001}) and RSA (Rivest-Shamir-Adleman algorithm~\cite{rsa_patent}) outperform MSS (common HBS, see Section~\ref{sec:mss}) in key generation, signature generation and verification time.
Therefore, finding possible improvements for existing HBS to make them more efficient is a crucial task and also the main goal of this work.
For related work on improving the performance of HBS in several ways, see Chapter~\ref{cha:stateOfTheArt}.

\section{Structure}
In Chapter~\ref{cha:background}, the fundamentals necessary for this work are introduced: After a brief introduction of digital signature schemes in general (see Section~\ref{sec:dig_sign_schemes}, the one-time signature schemes LD-OTS and W-OTS are proposed, including a basic example (see Section~\ref{sec:lamport_diffie_ots} and Section~\ref{sec:wots_general} respectively). Afterwards, the Merkle Signature Scheme (see Section~\ref{sec:mss}), Leighton-Micali Signature Scheme (see Section~\ref{sec:lms}) and eXtended Merkle Signature Scheme (see Section~\ref{sec:xmss}) are explained in detail. Chapter~\ref{cha:stateOfTheArt} summarizes state-of-the-art literature related to the thesis topic. 
The methods and ideas used for improving HBS performance in this work are proposed in Chapter~\ref{cha:methods}. Afterwards, the proposed methods and their results are applied in Chapter~\ref{cha:evaluation} 
Finally, Chapter~\ref{cha:conclusion} discusses the results of this work and provides possibilities for future work.
In Appendix~\ref{cha:appendix_t5_tree_implementation} and Appendix~\ref{cha:appendix2_performance_calc}, the \extree concept and generation of the evaluation results are implemented.