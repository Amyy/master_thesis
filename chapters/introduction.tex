\chapter{Introduction}
\label{cha:introduction}
Attack examples on RSA: ~\cite{RSA_pq-attack_examples_2018}, \cite{RSA_pq-attack_without_factorization_2018}
% kurz: warum es post-quantum gibt/es  notwendig ist, quantum-threat, shors algorithm/grovers algorithm -> good source: https://dl.acm.org/doi/fullHtml/10.1145/3398388
% shors algo: \cite{shors_algo_original_1999}

% was gibt es: isogeniebasiert, hashbasiert, lattice, code usw
The common classes of quantum-secure cryptographic systems are as follows:~\cite{book_pqc_bernstein_2004}
\begin{enumerate}
\item Hash-based cryptography
\item Code-based cryptography
\item Lattice-based cryptography
\item Multivariate-quadratic-equations cryptography
\item Secret-key cryptography
\end{enumerate}

\section{Motivation}
% why signature systems 
% PQC threat -> RSA not safe anmyore 

% performance PQC Algos -> worse than "classic" ones
Comparison RSA/ECDSA with WOTS/MSS:
"The results showed that the two classical algorithms perform better in terms of the efficiency in key generation time, signature generation and verification time."~\cite{comparison_performance_RSA_ECDSA_Merkle_WOTS_2021}

% -> improve performance of PQC algorithm

\section{Goals}
% Analyze state of the art 
% Finding possible improvements to existing hashbased signature systems

% T5 Paper hier bei Zielen -> Ziel ist Aufwand von Verfahren zu reduzieren, ohne Sicherheit zu beeinflussen

% Reference to Methods/T5 stuff etc.
The goal is to substitute the standard Merkle Tree with the \tftree and \extree concepts, to speed up LMS, XMSS and potentially other signature schemes based on Merkle Trees.