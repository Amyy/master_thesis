\chapter{Methods}
\label{cha:methods}
This chapter contains the methods used for extending the existing hash based digital signature systems in this work.

\section{T5 Hashing}
Dodis et al.~\cite{T5_paper} propose a method called $T5$ for hashing five inputs with three hash compression calls. The $5n$-to-$n$ compression function $T5$ (with $n$ being the hash digest length) is constructed out of $2n$-to-$n$ compression functions $h_1, h_2, h_3$:

\begin{equation}
\label{eq:t5_basic}
T_5(m_1, m_2, m_3, m_4, m_5) = h_3(h_1(m_1,m_2) \oplus m_5, h_2(m_3,m_4) \oplus m_5) \oplus m_5
\end{equation}
It is proven by Dodis et al.~\cite{T5_paper} that the $T5$ construction matches Stam’s bound~\cite{stams_bound2008}, providing $\tilde{\mathcal{O}}(q^2/2^n)$ collision security and $\mathcal{O}(q^3/2^{2n}+nq/2^n)$, $q \leq 2^{n/2}$ preimage security. It provides birthday security $\mathcal{O}(2^{n/2})$ (see also Section~\ref{sec:omit_coll_res}) for hashing 5 inputs using three $2n$-to-$n$ compression calls, instead of only 4 inputs (in prior constructions). For the full proof of collision resistance and preimage resistance of $T5$, see Section~4.1 and Section~4.2 in~\cite{T5_paper}. % what is q not sure
Therefore, $T5$ is improving the Merkle-D\aa mgard construction (with the initialization vector counted as message block) and Merkle trees by processing a fifth message block with the same number of compression function calls and basically the same level of collision security. 
For this work, the construction of $T5$ in combination with Merkle trees is of interest.  

\subsection{T5 Merkle Tree}
In Figure~\ref{img:t5_paper_block_depiction}, the construction of one \textit{T5-Block} out of a Merkle tree with height $d=2$ is shown:

\begin{itemize}
\item The variables  $m_1, m_2$ (respectively $m_3, m_4$) denote the left and right halves of input to compression function $h_1$ (respectively $h_2$).
\item The variable $a$ (respectively $b$) denotes the output of $h_1$ (respectively $h_2$).
\item The variables $c$ and $d$ denote the left and right halves of input to $h_3$.
\item The variable $e$ denotes the output of $h_3$.
\item The variable $f$ denotes the total output of $T5(m_1, m_2, m_3, m_4, m_5)$.
\end{itemize}
Therefore, the calculation of $T_5(m_1, m_2, m_3, m_4, m_5)$ consists of the following steps:
\begin{enumerate}
\item Calculation of the first layer of one $T_5$-Block (corresponds to compressing 4 leaves of a binary Merkle tree), two hash function calls $h_1, h_2$ are necessary:
\begin{align}
a = h_1(m_1, m_2) \\
b = h_2(m_3, m_4)
\end{align}

\item Calculation of the first $T_5$-specific intermediate step, adding $m_5$:
\begin{align}
c = a \oplus m_5 \\
d = b \oplus m_5
\end{align}

\item Calculation of the second layer of one $T_5$-Block (corresponds to compressing 2 nodes of a binary Merkle tree), one hash function calls $h_3$ is necessary:
\begin{align}
e = h_3(c,d)
\end{align}

\item Adding $m_5$ again (the second $T_5$ specific intermediate step) leads to the final result $f$:
\begin{align}
f = e \oplus m_5 = T_5(m_1, m_2, m_3, m_4, m_5)
\end{align}
\end{enumerate}

\begin{figure}
\centering
\includegraphics[]{images/Methods/abcd_paperT5_block_depiction.png}
\caption{Modified Merkle tree $T_5(m_1, m_2, m_3, m_4, m_5)$ of height d = 2,   with an extra input $m_5$ for the same 3 hash calls $h_1, h_2, h_3$. In this work, it is referred to as one $T_5$-Block.~\cite{T5_paper}}
\label{img:t5_paper_block_depiction}
\end{figure}

\subsection{T5 Openings}
To calculate the authentication path (see also Section~\ref{sec:mss_sig_gen}) in one T5-Block, two different approaches \textit{Conservative Opening} and \textit{Aggressive Opening} are shown by Dodis et al.~\cite{T5_paper}. 

\begin{figure}
\centering
\includegraphics[]{images/Methods/aggr_conserv_opening_T5.png}
\caption{Conservative and Aggressive Opening of one $T_5$-Block. The green variable denotes the current path value, the red variables are given in the authentication path for this $T_5$-Block.~\cite{T5_paper}.}
\label{img:t5_conserv_aggr_opening}
\end{figure}

\section{"My Idea"} % T5 + LMS
% describe "more aggressive" opening