\chapter{Related Work}
\label{cha:stateOfTheArt}

In this chapter, state-of-the-art works regarding \textit{hash-based signature systems (HBS)} are proposed. 
These HBS are separated into two categories: \textit{Stateful} hash-based signature schemes and \textit{stateless} hash-based signature schemes.  

\section{Stateful HBS}
Most common digital signature systems currently in use are stateless signature systems: The signer does not need to update the secret key, it is reused. 
the signer needs to maintain a dynamic state while signing messages. They do not fit the standard definition of a signature scheme in cryptography..~\cite{properties_stateless_HBS_2022}
The most recent stateful HBS include LMS and XMSS~\cite{stateful_hashbased_sign_schemes_NIST_2020}

% BPQS = blockchained post quantum signatures -> https://eprint.iacr.org/2018/658.pdf

\section{Stateless HBS}
The most common stateless HBS includes SPHINCS+~\cite{tweakable_basispaper_sphincs_2019}. 

% SPHINCS / SPHINCS+ [zustandslos/frei] (hypertrees) -> übersichtlich mit LMS/XMSS vergleichen
% wird "künstlich" stateless gemacht durch hypertrees -> unwahrscheinlich 2x gleichen one-time key zu bekommen
% ab hier -> versch. implementierungen / versionen von LMS, XMSS, SPHINCS+
% also (if exist): lattice based (RSA) ? -> check NIST competition 


%The Leighton-Micali Hash-Based Signature Scheme (LMS)~\cite{LMS_RFC8554} is based on the idea of Leighton and Micali~\cite{LMS_patent_leighton1995}. 
%As LMS is a major building block of this thesis, it is explained in detail in Chapter~\ref{cha:background}.-> add reference to LMS+XMSS section here, (when LMS section is created)

% https://eprint.iacr.org/2020/898.pdf -> read state of art here

% table has a little few information:
%\begin{table}
%\centering
%\begin{tabular}{l c} 
% \hline\noalign{\smallskip}
% \multicolumn{2}{c}{\textbf{Categories HBS}} \\
% \hline\noalign{\smallskip}
% Statful: & SPHINCS+  \\
% Stateless & LMS, XMSS \\
% \hline
%\end{tabular}
%\caption{The two categories in which hash-based signature schemes (HBS) are divided.} 
%\label{table:stateless_statful}
%\end{table}

% The focus will not be on works regarding solely other classical digital signature systems (e.g. RSA) as these generally outperform hash-based signature systems (performance, signature/key size), but do not fulfill the requirement to be quantum secure~\cite{RSA_pq-attack_examples_2018,comparison_performance_RSA_ECDSA_Merkle_WOTS_2021}.

% maybe add examples/cases where Merkle tree outperforms RSA/ECDSA. -> https://link.springer.com/chapter/10.1007/978-3-540-85893-5_8 , year 2008
% -> add attacks to RSA to introduction

