\chapter{State of the Art}
\label{cha:stateOfTheArt}

In this chapter, state-of-the-art concepts regarding \textit{hash-based signature systems (HBS)} are proposed. The focus will not be on works regarding solely other classical digital signature systems (e.g. RSA) as these generally outperform hash-based signature systems (performance, signature/key size), but do not fulfill the requirement to be quantum secure~\cite{RSA_pq-attack_examples_2018,comparison_performance_RSA_ECDSA_Merkle_WOTS_2021}.
% maybe add examples/cases where Merkle tree outperforms RSA/ECDSA. -> https://link.springer.com/chapter/10.1007/978-3-540-85893-5_8 , year 2008

% zustandsbasiert / zustandsfrei, kurze Übersicht 
% mention quantum insecure hash-based signature schemes
In general, hash-based signature schemes are separated into two categories: \textit{Stateful} hash-based signature schemes and \textit{stateless} hash-based signature schemes. The most recent stateful HBS include LMS and XMSS~\cite{stateful_hashbased_sign_schemes_NIST_2020}. The state-of-the-art stateless HBS is SPHINCS+~\cite{tweakable_basispaper_sphincs_2019}. % BPQS ?
% Nachteil stateless -> https://ietresearch.onlinelibrary.wiley.com/doi/epdf/10.1049/ise2.12040

% SPHINCS / SPHINCS+ [zustandslos/frei] (hypertrees) -> übersichtlich mit LMS/XMSS vergleichen
% ab hier -> versch. implementierungen / versionen von LMS, XMSS, SPHINCS+
% also (if exist): lattice based (RSA) ? -> check NIST competition 


%The Leighton-Micali Hash-Based Signature Scheme (LMS)~\cite{LMS_RFC8554} is based on the idea of Leighton and Micali~\cite{LMS_patent_leighton1995}. 
%As LMS is a major building block of this thesis, it is explained in detail in Chapter~\ref{cha:background}.-> add reference to LMS+XMSS section here, (when LMS section is created)





